\documentclass[letterpaper,11pt]{article}
\usepackage[utf8x]{inputenc}
\usepackage[spanish,mexico]{babel}
\usepackage{amsmath, amsthm,amssymb}
\usepackage[shortlabels]{enumitem}
\setenumerate{itemsep=0pt}
\usepackage[bottom=2.54cm, top=2.54cm,inner=2.54cm,outer=2.54cm]{geometry}
\usepackage{hyperref}
\hypersetup{colorlinks=true}
%\renewcommand\refname{Bibliografía}
\usepackage{listings}
%\usepackage[many]{tcolorbox}
%\tcbuselibrary{listings}
\lstset{frame=tb,
    showstringspaces=false,
    %columns=flexible,
    basicstyle={\ttfamily},
    numberstyle=\tiny\color{black},
    keywordstyle=\color{blue},
    commentstyle=\color{dkgreen},
    stringstyle=\color{mauve},
    breakindent=0pt,
    numbers=right,
    numbersep=-8pt,
    breaklines=true,
    breakatwhitespace=true,
    tabsize=3,
    emph={ },
    emphstyle={\color{red}},
    literate=
             {á}{{\'a}}1
             {é}{{\'e}}1
             {í}{{\'i}}1
             {ó}{{\'o}}1
             {ú}{{\'u}}1
             {ñ}{{\~n}}1
}

\begin{document}
\title{Proyecto de Aprendizaje Automatizado:\\
{\Large Un estudio sobre el uso del internet para comprar bienes}}
\author{Erick García Ramírez\footnote{
erick\_phy@ciencias.unam.mx, MCIC, IIMAS--UNAM.}}

\maketitle
\begin{abstract}
\noindent 
\end{abstract}

%\vspace{\baselineskip}
%{\noindent \bf Contenido}
%\begin{enumerate}
%    \item[1.] Introducción 
%    \item[2.] Antecedentes
%    \item[3.] Metodología
%    \item[4.] Resultados
%    \item[5.] Conclusiones
%    \item[6.] Referencias
%\end{enumerate}

\section{Introducción}
A lo largo de las últimas decadas las aplicaciones del Aprendizaje Automatizado han sido bastas y de
naturaleza muy diversa. Algunas de ellas, como lo son los Sistemas Automatizados de Recomendación y la Detección de
Operaciones Irregulares, tienen el objetivo de apoyar las actividades comerciales de empresas que ofertan
sus bienes y/o servicios por medio del internet. En este trabajo desarrollaremos una aplicación del Aprendizaje
Automatizado a este rubro.

Para una empresa que oferta algún producto o un servicio resulta muy valioso distinguir---de entre todas
las personas a las que dirige su publicidad---a aquellas personas que con mayor probabilidad comprarán o
contrarán su oferta. Determinar las características de los \emph{buenos clientes} y los \emph{malos clientes} es importante pues 
le a la empresa la oportunidad de optimizar sus estrategias de publicidad y de administración de clientes. 

Con la proliferación de grandes datos sobre el consumo de bienes y contratación de servicios, y en particular con la enorme
disponibilidad de datos de comercio en línea, las tareas de distinguir las características importantes de clientes es
una aplicación clásica y bien desarrollada de la Minería de Datos y el Aprendizaje Automatizado (\cite[Capítulo I]{berry}).  

\section{Antecedentes}
\section{Metodología}
\section{Resultados}
\section{Conclusiones}

\begin{thebibliography}{10}
    \bibitem{berry} Michael Berry y Gordon Linoff. {\em Data Mining Techniques For Marketing, Sales, and Customer Relationship Management.} 
        2da Edición. Wiley Publishing Inc., 2004. 
\end{thebibliography}

\end{document}
