\documentclass[letterpaper,11pt]{article}
\usepackage[utf8x]{inputenc}
\usepackage[spanish,mexico]{babel}
\usepackage{amsmath, amsthm,amssymb}
\usepackage[shortlabels]{enumitem}
\setenumerate{itemsep=0pt}
\usepackage{multicol}
\usepackage[bottom=2.2cm, top=2.1cm,inner=2.0cm,outer=1.8cm]{geometry}
\usepackage{hyperref}
\hypersetup{colorlinks=true}
\PassOptionsToPackage{hyphens}{url}\usepackage{hyperref}
\usepackage{graphicx}
\usepackage{abstract}
%\renewcommand\refname{Bibliografía}
\usepackage{listings}
%\usepackage[many]{tcolorbox}
%\tcbuselibrary{listings}
\lstset{frame=tb,
    showstringspaces=false,
    %columns=flexible,
    basicstyle={\ttfamily},
    numberstyle=\tiny\color{black},
    keywordstyle=\color{blue},
    commentstyle=\color{dkgreen},
    stringstyle=\color{mauve},
    breakindent=0pt,
    numbers=right,
    numbersep=-8pt,
    breaklines=true,
    breakatwhitespace=true,
    tabsize=3,
    emph={ },
    emphstyle={\color{red}},
    literate=
             {á}{{\'a}}1
             {é}{{\'e}}1
             {í}{{\'i}}1
             {ó}{{\'o}}1
             {ú}{{\'u}}1
             {ñ}{{\~n}}1
}

\begin{document}
\title{Proyecto de Aprendizaje Automatizado:\\
{\Large Un estudio sobre el uso del internet para comprar bienes}}
\author{Erick García Ramírez\footnote{
erick\_phy@ciencias.unam.mx, MCIC, IIMAS--UNAM.}}

\maketitle
\begin{abstract}
\noindent 
\end{abstract}

%\vspace{\baselineskip}
%{\noindent \bf Contenido}
%\begin{enumerate}
%    \item[1.] Introducción 
%    \item[2.] Antecedentes
%    \item[3.] Metodología
%    \item[4.] Resultados
%    \item[5.] Conclusiones
%    \item[6.] Referencias
%\end{enumerate}

\begin{multicols}{2}
\section{Introducción}
A lo largo de las últimas decadas las aplicaciones del Aprendizaje Automatizado han sido bastas y de
naturaleza muy diversa. Algunas de ellas, como lo son los Sistemas Automatizados de Recomendación y la Detección de
Operaciones Irregulares, tienen el objetivo de apoyar las actividades comerciales de empresas que ofertan
sus bienes y/o servicios por medio del internet. En este trabajo desarrollaremos una aplicación del Aprendizaje
Automatizado a este rubro.

Para una empresa que oferta algún producto o un servicio resulta muy valioso distinguir---de entre todas
las personas a las que dirige su publicidad---a aquellas personas que con mayor probabilidad comprarán o
contrarán su oferta. Determinar las características de los \emph{buenos clientes} y los \emph{malos clientes} es importante pues 
le a la empresa la oportunidad de optimizar sus estrategias de publicidad y de administración de clientes. 

Con la proliferación de grandes datos sobre el consumo de bienes y contratación de servicios, y en particular con la enorme
disponibilidad de datos de comercio en línea, las tareas de distinguir las características importantes de clientes es
una aplicación clásica y bien desarrollada de la Minería de Datos y el Aprendizaje Automatizado (\cite[Capítulo I]{berry}).  

En el presente trabajo aplicaremos técnicas de Aprendizaje Automatizado a los datos de venta de un producto por parte de una empresa, 
con el objetivo de descubrir las características de las personas que con mayor probabilidad comprarán el producto. 


\section{Antecedentes}
El uso de la Minería de Datos y el Aprendizaje Automatizado para apoyar actividades comerciales tiene un impacto
importante y una larga tradición, vea por ejemplo \cite{berry}\cite{big}\cite{tdk}.  Dentro de este contexto, el 
\emph{perfilado de usuarios o clientes} es una de las tareas en que dichas disciplinas sobresalen. En trabajos como
\cite{faw} y \cite{discov}.  

Un \emph{perfil} es un conjunto de información que funge como representación de una persona, usuario o cliente\cite{cufo}. La información que
conforma un perfil puede ser de naturaleza diversa; puede incluir, por ejemplo, datos conductuales, rasgos físicos y rasgos
socio-económicos. La información que debe constituir un perfil típicamente se determina a partir del uso que
se desea dar a dicho perfil. Por ejemplo, un perfil de un candidato a ingresar a alguna universidad
incluye información sobre su rendimiento en niveles acádemicos precedentes, su rendimiento en exámenes de ingreso y, 
posiblemente, algunos datos sobre su contexto socio-económico. En este caso la información necesaria para un perfil es
especificada por el cómite encargado del proceso de selección de la universidad. 

En muchos otros contextos especificar la información que debe constituir un perfil no es tan sencillo como en
el ejemplo del párrrafo anterior. 


El siguiente es el ejemplo que abordaremos en este trabajo. Cuando una empresa tiene la intención de lanzar un nuevo
producto a la venta, o de simplemente mejorar la estrategia publicitaria sobre algún producto, 
hay un gran interés en determinar qué personas serán sus potenciales compradores, es decir, es importante descubrir el perfil de los compradores 
del producto. El descubrimiento exitoso de dicho perfil permite a la empresa diseñar una estrategia de publicidad adecuada
para maximizar sus ventas y, en general, mejorar sus oportunidades para alcanzar sus metas comerciales.  Sin embargo, puede resultar
complicado determinar por adelantado y con precisión cuáles serán los rasgos del perfil que más influirán en la decisión
de compra del producto. 

Frente a tal problema, el punto de vista más común en la actualidad es el de colectar la mayor información posible para
formar un perfil provisional, para posteriormente analizar dichos datos y determinar entonces el perfil de cliente más
preciso posible (en este caso, el de las personas que con mayor probabilidad comprarán el producto). Este proceso se
lleva a cabo por medio de técnicas de Minería de Datos y Aprendizaje Automatizado. 


Entre las técnicas utilizadas para el perfilado de usuarios y/o clientes se encuentran los árboles de decisión,
los algoritmos de clustering y los  algoritmos de clasificación (vea \cite{mah}, donde  puede encontrar una tabla
    comprensiva sobre las técnicas usadas para esta tarea). Cada uno de estos tiene cualidades o desventajas que las hacen valiosas o
inapropiadas dependiendo del interés con el que se aborden. 

Los árboles de decisión tienen la característica de dar una buen panorama de las características del perfil, con
resultados interpretables de manera natural y de naturaleza cualitativa \cite[Capítulo 6]{berry}. Los algoritmos de clustering ofrecen
una segmentación de los usuarios/clientes, subrayando aquellas características que les hacen ser similares. En cambio,
los algoritmos de clasificación (e.g. máquinas de soporte vectorial y redes neuronales) ofrecen un gran poder de
predicción a costa de probablemente perder la interpretabilidad de los resultados, e.g.,\cite{chen}. En combinación, estas y otras
técnicas ayudan al investigador a definir el perfil. 

Trabajos
\section{Metodología}

El conjunto de datos que usaremos en este trabajo es \emph{Individual Company Sales Data}, disponible en www.kaggle.com~
\cite{kaggle}. Los datos corresponden a las ventas por internet de un producto por parte de una 
empresa\footnote{El conjunto se presenta anonimizado, no se conoce ni el producto ni el nombre de la empresa}. Cada
renglón en el conjunto corresponde a la información de un cliente.  Se colectan 14 datos sobre el cliente y una bandera
que señala si el cliente compró el producto o no. Después de remover aquellos renglones en los que hay algún valor
no conocido de algún atributo, el conjunto consta de 23558 muestras. La tabla~\ref{tab:1} describe la estructura y 
naturaleza del conjunto de datos.

%Comenzamos el trabajo sobre este conjunto de datos con un minado superficial de datos (describiremos a grandes rasgos
%la distribución de cada valor de los atributos). 
Las técnicas de Aprendizaje Automatizado aplicadas fueron: 

\begin{itemize}
    \item[(a)] Segmentación de los clientes por medio de clustering \cite{segmen}.
    \item[(b)]  Ajuste de un modelo por regressión lógistica. 
    \item[(c)] Ajuste de un modelo por máquinas de soporte vectorial.
    \item[(d)] Análisis de componentes principales PCA \cite{book2} (y posterior re-entrenamiento de (a)--(c)). 
\end{itemize}


Ya que la naturaleza de los datos es mixta, es decir, hay tanto datos númericos como categóricos,
debemos comentar sobre la preparación y el preprocesamiento de ellos. Se tomaron los siguientes pasos y consideraciones para el análisis. 

\begin{enumerate}[I.]
    \item En la columa 0, correspondiente a la bandera de si el cliente compró el producto o no, se cambió `Y' por 1
        y `N' por 0.
    \item La columna 3 correspondiente a la valuación de vivienda se re-escaló al intervalo [0,1]. 
    \item La columna 13 correspondiente a la probabilidad de comprar un auto nuevo se re-escaló a (0,1). 
    \item Se aplicó one-hot-encoding para sustituir todos aquellos valores categóricos por valores númericos. 
    \item En otra propuesta, antes de aplicar one-hot-encoding se cambiaron aquellos atributos de tipo categóricos
        ordinal (tipo \emph{likert}, como el nivel del ingreso familiar) por valores númericos re-escalados a (0,1).
        Posteriormente se aplicó one-hot-encoding a los atributos aún categóricos y se repitió (a)--(d).
\end{enumerate}
\end{multicols}

\begin{table}
\begin{center}
\scalebox{0.8}{
\begin{tabular}{|c|c|c|p{8cm}|}
    \hline
    \bf Columna & \bf Atributo & \bf Tipo de dato & \bf\hspace{2.5cm} Valores\\
    \hline 
    0 &\bf flag & categórico nominal &  Y, N si compró el producto, {\footnotesize(56.6\% y 43.4\% resp.)} \\
    \hline 
    1 &\bf  gender & categórico nominal & F, M  {\footnotesize (39\% y 61\% resp.)}\\
    \hline
    2 &\bf education & categórico ordinal & 0.lessHS, 1.HS, 2.SomeCollege, 3.Bach,
    4.Grad {\footnotesize (8.7\%, 21.4\%, 27.8\%, 25.4\% y 16.6\% resp.)}  \\ 
    \hline 
    3 &\bf house\_val & númerico & en [0, 999999]\\
    \hline 
    4 &\bf age & categórico nominal & 1\_Unk, 2upto25, 3upto35, 4upto45,
     5upto55, 6upto65, 7above65 {\footnotesize(13\%, 2.5\%, 10.7\%, 21.1\%, 25.8\%, 16.9\% y 9.8\% resp.)}\\ 
    \hline 
    5 &\bf online & categórico nominal &  Y, N el cliente tiene experiencia en compras por interne{\footnotesize (70.1\% y 29.9\% resp.)}; \\
    \hline
    6 &\bf  customer\_psy & categórico nominal & A--J, psicología del cliente, basaso en área de residencia  {\footnotesize (3.4\%, 21.9\%, 
            22.3\%, 5.6\%, 15.9\%, 9.3\%, 9.9\%, 1.6\%, 4.9\% y  5.1\% resp.)} \\
    \hline
    7 &\bf marriage & categórico nóminal & Y, N {\footnotesize (81.8\% y 18.2\% resp.)}\\
    \hline
    8 &\bf children & categórico nominal & Y, N, U {\footnotesize (31\%, 47\% y 21\% resp.)} \\
    \hline 
    9 &\bf occupation & categórico nominal & Professional, Blue Collar, Retired, SalesService,
    Others, Farm {\footnotesize (41.7\%, 16.9\%, 8.6\%, 28.2\%, 3.9\% y 0.7\% resp.)}\\
    \hline 
    10 &\bf  mortgage & categórico nominal & 1Low, 2Med,  3High {\footnotesize (69.9\%, 13.7\% y 16.3\% resp.)}\\
    \hline 
    11 &\bf house\_owner & categórico nominal & Owner, Renter {\footnotesize (79.7\% y 20.3\% resp.)}\\
    \hline 
    12 &\bf region & categórico nominal & West, South, Midwest, Northeast, Rest {\footnotesize (21.8\%, 39\%, 20.7\%,
    17.8\% y 0.6\% resp.)}\\
    \hline 
    13 &\bf car\_prob & categórico ordinal & 1--9; probabilidad de que comprará un auto nuevo {\footnotesize (32.1\%,
    18.8\%, 13.4\%, 7.1\%, 6.6\%, 5\%, 4.8\%, 6.3\% y 5.8\% resp.)}\\
    \hline 
    14 &\bf fam\_income & categórico ordinal & A--L; nivel del ingreso familiar, L es el más alto {\footnotesize (5\%,
    4.8\%, 5.7\%, 10.5\%, 20.7\%, 17.4\%, 11\%, 6.8\%, 4.6\%, 4.6\%, 4.1\% y 4.6\% resp.)}\\
    \hline
\end{tabular}}
\end{center}
\caption{\small La estructura de la base de datos.}
\label{tab:1}
\end{table}

\begin{multicols}{2}
\section{Resultados}

\subsection*{Clustering} Aplicamos el algoritmo K-means~\cite[Capítulo 13]{book2}.
\section{Conclusiones}

\begin{thebibliography}{10}
    \bibitem{berry} M. Berry y G. Linoff. {\em Data Mining Techniques For Marketing, Sales, and Customer Relationship Management.} 
        2da Edición. Wiley Publishing Inc., 2004. 
    \bibitem{big} J. Dean. {\em Big Data, Data Mining, and Machine Learning: Value Creation for Business, Leaders and
        Practitioners}. John Wiley  Sons, 2014.
    \bibitem{tdk} TDK-technologies. {\em Business Intelligence, Data Mining and Machine Learning.} 
        Disponible en: \url{www.tdktech.com/tech-talks/business-intelligence-data-mining%-machine-learning}. 
        \bibitem{cufo} A. Cufoglu. {\em User Profiling-A Short Review}. International Journal of Computer Applications,
        Volume 108-- No. 3, 2014.
    \bibitem{faw} T. Fawcett y P. Foster. {\em Combining Data Mining and Machine Learning for Effective User
        Profiling.} En KDD-96 Proceedings, págs. 8-13, AAAI, 1996.
    \bibitem{discov} A. Bellogín, I. Cantador, P. Castells y Á. Ortigosa. {\em Discovering Relevant Preferences in a Personalised
        Recommender System using Machine Learning Techniques}, en Preference Learning Workshop, en la 8th
        European Conference on Machine  Learning and Principles and Practice of Knowledge Discovery in Databases, 2008. 
    \bibitem{mah} M. Harandi. {\em User Profiling in News Recommender Systems}. Disponible en:
        \url{https://www.ntnu.no/wiki/download/attachments/86731314/User%20Profiling%20NRS.pdf?version%3}
    \bibitem{chen} Q. Chen, A. Norcio y J. Wang. {\em Neural Network Based Stereotyping for User Profiles}. Neural
        Computing and Applications No. 9, págs. 259--265,  2000.
    \bibitem{kaggle} {\em Individual Company Sales Data.} Disponible en:
        \url{https://www.kaggle.com/mickey1968/individual-company-sales-data}. 
    \bibitem{segmen} K. Kashwan y C. Velu. {\em Customer Segmentation Using Clustering and Data Mining Techniques}. International Journal of Computer Theory 
        and Engineering, Vol. 5 (6), págs. 856--860, 2013. 
    \bibitem{book2} M. Zaki y W. Meira. {\em Data minining and Analysis}. Cambridge University Press, 2014. 

\end{thebibliography}
\end{multicols}
\end{document}

