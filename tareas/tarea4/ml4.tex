\documentclass[letterpaper,11pt]{article}
\usepackage[utf8x]{inputenc}
\usepackage[bottom=1.8cm,top=1.7cm,left=1.2cm,right=1.4cm]{geometry}
\usepackage[spanish,mexico]{babel}
\usepackage[many]{tcolorbox}
\tcbuselibrary{listings}
\usepackage{amsmath,amssymb,amsthm}
\usepackage[shortlabels]{enumitem}
\usepackage{xspace}
\usepackage{color}
\usepackage{tikz}
\usetikzlibrary{arrows,automata}
\usepackage{multicol}
\usepackage{vwcol}
\usepackage{txfonts}
\usepackage{tabularx}

\author{Erick García Ramírez}
\begin{document}
\begin{flushleft}
    {\bf Tarea 4: Modelos gráficos probabilísticos}\\
    {\bf Aprendizaje Automatizado}\\
    {MCIC, 2019-II}\\
    03 de Mayo de 2019
\end{flushleft}
\vspace{-6\baselineskip}
\begin{flushright}
    {\bf Alumno:} Erick García Ramírez\\
     erick\_phy@ciencias.unam.mx\\
\end{flushright}
\vspace{2\baselineskip}
\subsection*{Problema 1}
\begin{itemize}
    \item Proponer tres modelos para el sistema. 
        \begin{multicols}{2}
     Modelo (I)
\begin{center}
    \begin{tikzpicture}[->,>=stealth',shorten >=1pt,auto,node distance=2cm,
        scale = 1,transform shape]

  \node[state] (T) {$T$};
  \node[state] (I) [below of=T] {$I$};
  \node[state] (D_I) [above right of=I, xshift=1cm] {$D_I$};
  \node[state] (A) [below of=I] {$A$};
  \node[state] (D_A) [above left of=A,xshift=-0.5cm] {$D_A$};

  \path (T) edge              node {$$} (I)
        (D_I) edge              node {$$} (I)
        (D_A) edge              node {$$} (A)
        (I) edge              node {$$} (A);
\end{tikzpicture}

Distribución conjunta de la red: \[P(T)P(D_A)P(D_I)P(I|T,D_I),P(A|D_A,I)\]
    \vfill\null
    \columnbreak
\end{center}
\noindent En el modelo (I):
    \begin{itemize}[$\bullet$]
        \item La temperatura, si la alarma es defectuosa y si el indicador es defectuoso son variables no
            condicionadas. 
        \item Que la alarma se active ($A=1$) está condicionada al valor del indicador y de si la alarma
            es defectuosa. 
        \item El valor del indicador está condicionado a la temperatura y a que el indicador sea defectuoso. 
        \item Se tiene el mayor número de independencias condicionales. Este modelo tiene el menor número de arcos de
            una red que aún captura el comportamiento del sistema; quitar un arista revoca alguna de las funcionalidades
            importantes del sistema.
    \end{itemize}
\end{multicols}
\vspace{\baselineskip}
\begin{multicols}{2}
     Modelo (II) 
\begin{center}
    \begin{tikzpicture}[->,>=stealth',shorten >=1pt,auto,node distance=2cm,
            scale = 1,transform shape]
      \node[state] (T) {$T$};
      \node[state] (I) [below of=T] {$I$};
      \node[state] (D_I) [above right of=I,xshift=1cm] {$D_I$};
      \node[state] (A) [below of=I] {$A$};
      \node[state] (D_A) [above left of=A, xshift=-0.5cm] {$D_A$};

      \path (T) edge              node {$$} (I)
            (T) edge              node {$$} (D_I)
            (T) edge              node {$$} (D_A)
            (D_I) edge              node {$$} (I)
            (D_A) edge              node {$$} (A)
            (I) edge              node {$$} (A);
    \end{tikzpicture}

Distribución conjunta de la red: \[P(T)P(D_A|T)P(D_I|T)P(I|T,D_I),P(A|D_A,I)\]
    \vfill\null
    \columnbreak
\end{center}
En el modelo (II):
    \begin{itemize}[$\bullet$]
        \item Ahora se representa que el hecho de que la alarma y/o el indicador sean defectuosos
             depende directamente de la temperatura. 
        \item Se eliminan algunas independencias condicionales. E.g., en el Modelo (I) se tiene que $T\Perp A|I$, pero
            ahora en el Modelo (II) el camino $TD_AA$ es activo y por lo tanto aquella independecia condicional ya no
            se cumple. 
    \end{itemize}
\end{multicols}
\newpage
\begin{multicols}{2}
     Modelo (III)
\begin{center}
    \begin{tikzpicture}[->,>=stealth',shorten >=1pt,auto,node distance=2cm,
            scale = 1,transform shape]

      \node[state] (T) {$T$};
      \node[state] (I) [below of=T] {$I$};
      \node[state] (D_I) [above right of=I,xshift=1cm] {$D_I$};
      \node[state] (A) [below of=I] {$A$};
      \node[state] (D_A) [above left of=A, xshift=-0.5cm] {$D_A$};

      \path (T) edge              node {$$} (I)
            (T) edge              node {$$} (D_I)
            (T) edge              node {$$} (D_A)
            (D_I) edge              [bend left=40] node {$$} (D_A)
            (D_I) edge              node {$$} (I)
            (D_A) edge              node {$$} (A)
            (T) edge                [bend right=80,looseness=2.4] node {$$} (A)
            (D_I) edge                [bend left=30] node {$$} (A)
            (I) edge              node {$$} (A);
    \end{tikzpicture}

    Distribución conjunta de la red: \[P(T)P(D_A|T,D_I)P(D_I|T)P(I|T,D_I),P(A|T,D_A,D_I,I)\]
    \vfill\null
    \columnbreak
\end{center}
En el modelo (III):
    \begin{itemize}[$\bullet$]
        \item  E.g., ahora se representa que el hecho de que la alarma sea defectuosa depende directamente de que el
            indicador lo sea. 
        \item Se eliminan aún más independencias condicionales; e.g., en los modelos (I) y (II) se cumple que $D_I|\Perp
            D_A|T$, sin embargo esto no se cumple en el Modelo (III) pues se tiene explícita la dependencia de 
            $D_A$ con respecto a $D_I$.
    \end{itemize}
\end{multicols}
    \item Suponemos que las variables I y T pueden tomar un máximo de 100 valores. La siguiente tabla muestra
        cuántos valores se requieren conocer para calcular la probabilidad de los valores en cada nodo en cada modelo.
        NVR abrevia ``Número de valores requeridos''.
        %¿cuál sería el número de valores necesarios en cada nodo y el total en cada una de las redes?

        \begin{center}
        \begin{tabular}{|c|r|r|r|}%{|l{2cm} |p{4.4cm} |p{4.4cm} |p{4.4cm}|}
            \hline
            Variable & NVR en Modelo (I) & NVR en Modelo (II) & NVR
            en Modelo (III) \\ 
            \hline
            T & 0 & 0 & 0 \\
            \hline
            $D_I$ & 0 & 100 & 100\\
            \hline
            $D_A$ & 0 & 100& 2*100 = 200 \\
            \hline
            I & 2*100 = 200 & 2*100 = 200 & 2*100 = 200\\
            \hline
            A & 2*100 = 200 & 2*100 = 200 & 100*2*100*2= 40000\\
            \hline
            \textbf{Total}& 400 &  600 & 40500 \\
            \hline
        \end{tabular}
    \end{center}

        El número de valores requeridos para la variable A en el Modelo (III) es muy grande ya que en este modelo
        sugerimos que los valores de dicha variable dependen directamente de los de T (100 valores), $D_A$ (2 valores),
        I (100 valores) y
        $D_I$ (2 valores), lo que resulta en que se requieren conocer $100*2*100*2=40000$ valores para conocer las
        probabilidades de los valores de A.
    %    \begin{center}
    %    \begin{tabular}{|c|r|r|r|}%{|l{2cm} |p{4.4cm} |p{4.4cm} |p{4.4cm}|}
    %        \hline
    %        Variable & \# de valores en Modelo (I) & \# de valores en Modelo (II) & \# de valores
    %        en Modelo (III) \\ 
    %        \hline
    %        T & 100 & 100 & 100 \\
    %        \hline
    %        $D_I$ & 2 & 2*100 = 200& 2*2 = 4\\
    %        \hline
    %        $D_A$ & 2 & 2*100 = 200& 2*100 = 200 \\
    %        \hline
    %        I & 100*(2*100) = 20000 & 100*(2*100) = 20000 & 100*(2*100) = 20000\\
    %        \hline
    %        A & 2*(2*100) = 400 & 2*(2*100) = 400 & 2*100*100*2= 40000\\
    %        \hline
    %        \textbf{Total} & & & \\
    %        \hline
    %    \end{tabular}
    %\end{center}

\end{itemize}

\subsection*{Problema 2}
\begin{multicols}{2}
    \columnsep=3cm
\begin{center}
    \hspace{1cm}
\begin{tikzpicture}[->,>=stealth',shorten >=1pt,auto,node distance=2cm,
        scale = 1,transform shape]

  \node[state] (A) [] {$A$};
  \node[state] (T) [below of=A] {$T$};
  \node[state] (O) [xshift=0.4cm, below right of=T] {$O$};
  \node[state] (X) [xshift=-0.4cm,below left of=O] {$X$};
  \node[state] (C) [xshift=0.4cm, above right of=O] {$C$};
  \node[state] (F) [xshift=0.4cm, above right of=C] {$F$};
  \node[state] (B) [xshift=0.4cm, below right of=F] {$B$};
  \node[state] (D) [xshift=0.4cm, below right of=O] {$D$};

  \path (A) edge              node {$$} (T)
        (T) edge              node {$$} (O)
        (O) edge              node {$$} (X)
        (C) edge              node {$$} (O)
        (O) edge              node {$$} (D)
        (F) edge              node {$$} (C)
        (F) edge              node {$$} (B)
        (B) edge              node {$$} (D);

\end{tikzpicture}
\end{center}
\columnbreak
\vspace*{1.5\baselineskip}
\hspace*{1cm}
{\small
\begin{tabular}{c c}
Etiqueta & Significado \\ 
\hline
A        & Visita a Asia\\
T        & Tuberculosis\\
C        & Cáncer de pulmón\\
O        & Tuberculosis o Cáncer de pulmón\\
F        & Fumador\\
X        & Prueba de rayos X positiva\\
B        & Bronquitis\\
D        & Disnea
\end{tabular}
}
\end{multicols}

Recordemos que $X\Perp Y|Z$ si todos los caminos de $X$ a $Y$ son inactivos. Un camino de $X$ a $Y$ es
        inactivo si contiene una tripleta inactiva; la tripleta $ABC$ es inactiva si: (a) $B$ está en $Z$ (i.e. $B$ es
        observado) y $B$ es head-to-tail o tail-to-tail, o (b) $B$ es head-to-head y ningún descendiente de $B$,
        incluyendo B,  está en $Z$.
\begin{enumerate}[a.]
    \item $T\Perp F|D$. \\
        Afirmamos que no ocurre que $T\Perp F|D$. Se tienen dos caminos de T a F, TOCF y TODBF. Afirmamos que TOCF es activo.
        Dicho camino contiene dos tripletas,
        TOC y OCF. En TOC, O es head-to-head pero tiene como descendiente al nodo observado D. Por otro lado, en OCF, C
        es head-to-tail pero C no es observado. Por lo tanto, TOCF no contiene ninguna tripleta inactiva. 
    \item $C\Perp B|F$.\\
        Afirmamos que sí ocurre que $C\Perp B|F$. Hay dos caminos de C a B, CFB y  CODB. Veamos que ambos son inactivos. CFB consiste de una sola tripleta, la
        cual es inactiva ya que F es tail-to-tail y F es observado. Para CODB tenemos las tripletas COD y ODB. La
        segunda de éstas es inactiva pues D es head to head y ni D ni ninguno de sus descendientes es observado. 
    \item $A\Perp F|C$.\\
        Afirmamos que sí ocurre que $A\Perp F|C$. Se tienen dos caminos de A a F, ATOCF y ATODBF. ATOCF es inactivo pues en la tripleta  OCF, C es head-to-tail y
        observado. También ATODBF es inactivo pues en la tripleta ODB, D es head-to-head, no observado y ninguno de sus
        descendientes es observado. Por lo tanto, ambos caminos son inactivo. 
    \item $A\Perp F|C,D$. \\
        Afirmamos que no ocurre que $A\Perp F|C,D$. Como  en el inciso anterior, sólo hay que considerar los  caminos ATOCF y
        ATODBF. El primero de ellos sigue siendo inactivo, sin embargo, ahora ATODBF es activo. Ninguna de las tripletas
        ATO, TOD, ODB y DBF es inactiva. En el inciso anterior ODB era inactiva, pero en este caso D es head-to-head
        pero D sí es observado. Así, ODB es activa.
\end{enumerate}

\subsection*{Problema 3}
\begin{multicols}{2}
    \columnsep=3cm
\begin{center}
    \hspace{1cm}
\begin{tikzpicture}[->,>=stealth',shorten >=1pt,auto,node distance=2cm,
        scale = 1,transform shape]

  \node[state] (E) [xshift=0.4cm] {$E$};
  \node[state] (F) [xshift=-0.4cm, below left of=E] {$F$};
  \node[state] (S) [xshift=0.4cm, below right of=E] {$S$};
  \node[state] (V) [xshift=0.4cm, below right of=F] {$V$};
  \node[state] (C) [xshift=0.4cm, below right of=S] {$C$};
  \node[state] (D) [below left of=V] {$D$};

  \path (E) edge              node {$$} (S)
        (E) edge              node {$$} (F)
        (F) edge              node {$$} (V)
        (S) edge              node {$$} (V)
        (V) edge              node {$$} (D)
        (S) edge              node {$$} (C);

\end{tikzpicture}
\end{center}
\columnbreak
\vspace*{1.5\baselineskip}
\hspace*{1cm}
{\small
\begin{tabular}{c c}
Etiqueta & Significado \\ 
\hline
E        & Ébola \\
F        & Fiebre\\
S        & Sangrado\\
V        & Visita clínica\\
C        & Complicaciones\\
D        & Ve a especialista\\
\end{tabular}
}
\end{multicols}
\begin{multicols}{2}
    \columnsep=3cm
    \begin{itemize}
\item P (E = verdadero) = 0.01
\item P (F = verdadero$\,|\,$E = sí) = 0.6
\item P (F = verdadero$\,|\,$E = falso) = 0.1
\item P (S = verdadero$\,|\,$E = verdadero) = 0.8
\item P (S = verdadero$\,|\,$E = falso) = 0.05
\item P (V = verdadero$\,|\,$F = verdadero, S = verdadero) = 0.8
\item P (V = verdadero$\,|\,$F = verdadero, S = falso) = 0.5
\item P (V = verdadero$\,|\,$F = falso, B = verdadero) = 0.7
\item P (V = verdadero$\,|\,$F = falso, B = falso) = 0.0
\item P (C = verdadero$\,|\,$S = verdadero) = 0.75
\item P (C = verdadero$\,|\,$S = falso) = 0.1
\item P (D = verdadero$\,|\,$V = verdadero) = 0.6
\item P (D = verdadero$\,|\,$V = falso) = 0.0
    \end{itemize}
\end{multicols}
\begin{enumerate}[a.]
    \item Dados los valores arriba, la distribución conjunta de la red es:
        \[P(E,F,S,V,D,C)=P(E)P(F|E)P(S|E)P(V|F,S)P(D|V)P(C|S)= (0.01)(0.6)(0.8)(0.8)(0.6)(0.75)=0.001728\]
    \item Si un paciente es llevado al doctor (D = verdadero), usando pgmpy obtenemos que la probabilidad de que no
        tenga ébola es 
        \[P \left(\text{Ebola = falso}\,|\,\text{Visita especialista = verdadero}\right)=0.0013.\]
    (Vea el código adjunto en Problema3\_t4.ipynb).    

\item Convertimos el modelo anterior a un campo aleatorio de Markov.

    \textbf{Paso1: Moralización}. La gráfica moral resulta de agregar aristas (no-dirigidas) entre cada par de padres de
    cualquier nodo, y eliminar la dirección de todas las aristas. En este caso sólo se agrega una arista entre F y S, que son padres de V. La gráfica moral es:

    \begin{center}
        \hspace{1cm}
    \begin{tikzpicture}[-,>=stealth',shorten >=1pt,auto,node distance=2cm,
            scale = 1,transform shape]

      \node[state] (E) [xshift=0.4cm] {$E$};
      \node[state] (F) [xshift=-0.4cm, below left of=E] {$F$};
      \node[state] (S) [xshift=0.4cm, below right of=E] {$S$};
      \node[state] (V) [xshift=0.4cm, below right of=F] {$V$};
      \node[state] (C) [xshift=0.4cm, below right of=S] {$C$};
      \node[state] (D) [below left of=V] {$D$};

      \path (E) edge              node {$$} (S)
            (E) edge              node {$$} (F)
            (F) edge              node {$$} (S)
            (F) edge              node {$$} (V)
            (S) edge              node {$$} (V)
            (V) edge              node {$$} (D)
            (S) edge              node {$$} (C);

    \end{tikzpicture}
    \end{center}

    \textbf{Paso 2: Completar cliques}. Asegurar que para cada factor del tipo $P(v|x_1,\dots ,x_m)$ en la red
    bayesiana, el conjunto $\{v,x_1,\dots, x_m$\} es ahora parte de un clique en la gráfica no dirigida.

    En nuestro caso debemos considerar los factores $P(F|E)$, $P(S|E)$, $P(V|F,S)$, $P(D|V)$, y $P(C|S)$. Por lo
    tanto, tenemos que asegurar que cada uno de los siguientes conjuntos  está contenido en un clique: $\{F,E\}$,
    $\{S,E\}$, $\{V,F,S\}$, $\{D,V\}$ y $\{C,S\}$. Se puede ver que esto ya se cumple para la gráfica arriba
    mostrada, y, por lo tanto, esa es la gráfica dirigida buscada. 

    \item Dada la nueva información, la probabilidad $P(V=\text{verdadero}\,|\,F=\text{verdadero})$ ha aumentado. 
    \begin{itemize}
        \item ¿Qué probabilidades condicionales en la red se modifican debido a este cambio y en qué sentido?\\
            Como ya mencionamos, $P(V|F)$ aumenta (ahora es más probable que una persona con fiebre visite la clínica).
            En general, las probabilidades condicionales que cambiaran son de la forma $P(X|Z)$ siempre que no sea
            cierto que $X\Perp F|Z$.  
        \item Describe cualquier efecto que esto tenga en la proporción de personas con complicaciones
        que visiten la clínica. Menciona exactamente qué probabilidades condicionales usaste para
        llegar a tu conclusión.
    \end{itemize}
    \item Asume que alguien que no tiene fiebre va al doctor, ¿qué relación de independencia condicional
existe en la distribución que no puede ser descubierta a través del grafo solamente?
\end{enumerate}
\end{document}
