\documentclass[letterpaper,11pt]{article}
\usepackage[utf8x]{inputenc}
\usepackage[bottom=1.4cm,top=1.4cm,left=1cm,right=1.3cm]{geometry}
\usepackage[spanish,mexico]{babel}
\usepackage[many]{tcolorbox}
\tcbuselibrary{listings}
\usepackage{amsmath,amssymb,amsthm}
\usepackage[shortlabels]{enumitem}
\usepackage{xspace}
\usepackage{color}

\author{Erick García Ramírez}
\begin{document}
\begin{flushleft}
    {\bf Tarea 1: Recordatorio de probabilidad}\\
    {\bf Aprendizaje Automatizado}\\
    {MCIC, 2019-II}\\
    20 de Febrero de 2019
\end{flushleft}
\vspace{-6\baselineskip}
\begin{flushright}
    {\bf Alumno:} Erick García Ramírez\\
     erick\_phy@ciencias.unam.mx\\
\end{flushright}
\vspace{2\baselineskip}
\begin{enumerate}
\item  \emph{Se usa el siguiente proceso aleatorio para meter 2 pelotas en una caja: se tira una moneda y
se mete una pelota roja si sale águila o azul si sale sol. Posteriormente de esta caja se sacan
repetidamente de forma aleatoria con reemplazo 3 pelotas, las cuales resultan rojas. ¿Cuál es
la probabilidad de que las 2 pelotas de la caja sean rojas?}\\
{\bf Soln.}  Sean $A$ el evento ``las dos pelotas en la caja son rojas'' y $T$ el evento ``las tres pelotas
sacadas son rojas''. Buscamos $P(A|T)$. Note que $P(T|A)=1$ pues si ambas pelotas son rojas necesariamente las
tres pelotas sacadas debieron ser rojas. También, por el proceso seguido para poner las pelotas en la caja
(donde suponemos que la moneda usada es justa), sabemos que $P(A)=1/4$. Observe que si no ocurrió $A$, la 
probabilidad de haber sacado tres veces una pelota roja es $(1/2)^3(2/3)$, es decir, $P(T|A^c)=1/12$.  
Como último ingrediente notemos que por el teorema de la probabilidad total
$P(T)=P(T|A)P(A)+P(T|A^c)P(A^c)=1\cdot 1/4+1/12\cdot 3/4=5/16$. Concluimos que:
\[P(A|T)=\frac{P(T|A)P(A)}{P(T)}=\frac{1\cdot\frac{1}{4}}{\frac{5}{16}}=\frac{4}{5}.\]

\item \emph{En una caja hay 3 playeras rojas y 5 verdes de talla grande, además de 2 playeras rojas y
5 verdes de talla chica. Se saca de forma aleatoria y uniforme una playera de dicha caja y
resulta ser roja. ¿Cuál es la probabilidad de que sea de talla grande?}\\
{\noindent \bf Soln:} Sea $G$ el evento ``la playera sacada es de talla grande'' y $R$ el evento ``la playera
sacada es roja''. Así, buscamos determinar $P(G|R)$, la probabilidad de que  ``la playera sacada es
grande dado que es roja''. La
probabilidad sencilla de $R$ es $P(R)=5/15$, pues hay 5 playeras rojas entre las 15 totales. Por otro
lado, la probabilidad de $G\cap R$  es $P(G\cap R)=3/15$, pues hay 3 playeras rojas y grandes
entre las 15 playeras totales.
Por lo tanto tenemos que: 
\[P(G| R)=\frac{P(G\cap R)}{P(R)}=\frac{3}{5}.\]
\item \emph{Un aeropuerto cuenta con un sistema que es capaz de identificar correctamente si una persona
es terrorista el 95\% de las veces y si una persona es un ciudadano honrado también el 95\% de
las veces. Un informante alerta a las autoridades sobre la presencia de exactamente 1 terrorista
en un avión con 100 pasajeros, por lo que las autoridades detienen al primer pasajero y el
sistema detecta que es terrorista. ¿Cuál es la probabilidad de que esta persona realmente sea
terrorista?}\\
    {\bf Soln.} Sean $S$ el evento ``el sistema identificó a la persona como terrorista'', $T$ el  evento ``la
    persona es un terrorista'' y $NT$ el evento ``la persona no es un terrorista'' (i.e. $NT= T^c$). Deseamos 
    conocer $P(T|S)$, la probabilidad de que la persona sea terrorista dado que el sistema la identificó como
    uno. Sabemos
    que $P(S|T)=0.95$, pues el sistema identifica como terrorista a una persona que en verdad es terrorista el
    95\% de las veces. También, $P(S|NT)=1-0.95=0.05$ pues el sistema identifica como terrorista a una persona
    que no lo es (i.e. se equivoca) en  5\% de las veces. Como exactamente una persona entre los 100 pasajeros
    es un terrorista, tenemos que $P(T)=1/100=0.01$ y que $P(NT)=1-P(T)=0.99$. 

    Por lo tanto, 
    \[P(T|S)=\frac{P(S|T)P(T)}{P(S|T)P(T)+P(S|NT)P(NT)} = \frac{(0.95)(0.01)}{(0.95)(0.01)+(0.05)(0.99)}=
    0.1616.\]
\item \emph{Un paciente obtiene un resultado positivo en una prueba de una enfermedad muy seria. Esta
prueba es muy precisa: la probabilidad de que la prueba sea correcta (positivo o negativo) es
de 0.99. Sin embargo, la enfermedad es extremadamente rara y sólo afecta a 1 de cada 10000
personas. ¿Cuál es la probabilidad de que el paciente realmente tenga la enfermedad?}\\
    {\bf Soln.} Sean $S$ el evento ``el resultado de la prueba es positivo'', $E$ el evento ``la persona tiene la
    enfermedad'' y $NE$ el evento ``la persona no tiene la enfermedad''. Buscamos calcular $P(E|S)$. Sabemos que
    la probabilidad de que la prueba dé un resultado positivo dado que la persona en verdad tiene la
    enfermedad es de 0.99, i.e. 
    $P(S|E)=0.99$. Como la prueba tiene una probabilidad de 0.01 de equivocarse, también sabemos que
    $P(S|NE)=0.01$. Por último, la probabilidad general de que la persona tenga la enfermedad es
    $P(E)=1/10000=0.0001$, y en consecuencia $P(NE)=1-P(E)=0.9999$.  Por lo tanto tenemos que: 
    \[P(E|S)=\frac{P(S|E)P(E)}{P(S|E)P(E)+P(S|NE)P(NE)} = \frac{(0.99)(0.0001)}{(0.99)(0.0001)+(0.01)
    (0.9999)}=0.0098\]

\item \emph{Juan y Pedro tiran una moneda cada uno. Juan apuesta que ambas monedas caerán iguales
y Pedro que caerán diferentes. Prueba que incluso si una de las monedas estuviera cargada, el juego
seguiría siendo justo.}\\
    {\bf Soln.} Supongamos que la moneda de Juan es justa y que la moneda de Pedro prodría no ser justa,
    teniendo como probabilidad de obtener $S$ (sol) $p\in  [0,1]$. En consecuencia, la
    probabilidad de que caiga $A$ (águila) al lanzar la moneda de Pedro es $(1-p)$. Al lanzar las dos 
    monedas obtenemos el espacio muestral $\{SS,SA,AS,SS\}$; el evento que favorece a Juan es 
    $X:=\{SS,AA\}$ y el que favorece a Pedro es $Y:=\{SA,AS\}$. Deseamos mostrar que $P(X)=P(Y)$.
    Por la definición de $p$ arriba tenemos que $P(X)=P(SS)+P(AA)=\frac{1}{2}p+\frac{1}{2}(1-p)$ y
    $P(Y)= \frac{1}{2}(1-p)+\frac{1}{2}p$. Claramente, $P(X)=\frac{1}{2}=P(Y)$. \hfill $\blacksquare$
\item \emph{Un estudiante debe elegir una de las siguientes materias: matemáticas, física o química. Es
igualmente probable que elija matemáticas o física y doblemente probable que elija química.
Calcula las probabilidades de cada materia.}\\
    {\bf Soln.} Sea $M$ el evento ``el estudiante elige matemáticas''. Similarmente se definen los eventos
    $F$ y $Q$. Buscamos $P(M)$, $P(F)$ y $P(Q)$. Como el estudiande debe tomar alguna de 
    las tres materias tenemos que $P(M\cup F\cup Q)=1$. Más aún, los eventos son disjuntos\footnote{Suponemos
    que el enunciado del problema establece que el estudiante debe tomar exactamente una materia.}, 
    así que $1=P(M)+P(F)+P(Q)$. Usando que $P(M)=P(F)$ y que $P(Q)=2P(M)$, tenemos que $1=4P(M)$.
    Concluimos que $P(M)=P(F)=1/4$ y $P(Q)=1/2$. 
\item \emph{En un concurso de TV el anfitrión le da a un concursante 3 puertas a elegir. Detrás de 2 de
estas puertas hay una cabra y en la restante un auto. El concursante elige la puerta 1 y el
anfitrión descubre la puerta 3, en la cual hay una cabra. Después de descubrir la puerta 2 el
anfitrión le da la opción al concursante de cambiar la puerta 1 que había elegido originalmente
por la puerta 2 aún sin descubrir. ¿Existe alguna diferencia si el concursante cambia de puerta
2? Explica tu respuesta.}\\
    {\bf Soln.} Sí hay una diferencia. Bajo la situación descrita, si el concursante decide cambiar 
    su elección, la probabibilidad de haber elegido la puerta con el auto detrás es más alta que si no cambia. 

    Sean $A_i$ el evento ``el auto está detrás de la puerta $i$", $E_i$ el evento ``el concursante elige la
    puerta $i$'' y $P_i$ el evento ``el anfitrión abre la puerta $i$''. Deseamos mostrar que
    $P(A_1|E_1,P_3)<P(A_2|E_2,P_3)$.

    Notemos que $P(P_3|A_2,E_1)=1$ pues si el auto está detrás de la puerta 2 y el concursante elijió la
    puerta 1, entonces al anfitrión no le queda más que abrir la puerta 3. Tenemos que:
    \begin{align*}
    P(A_2|E_1,P_3)=& \frac{P(A_2\cap E_1\cap P_3)}{P(E_1,P_3)}=\frac{P(P_3|E_1,A_2)P(E_1,A_2)}{P(E_1,P_3)}=\\
    =& \frac{P(E_1,A_2)}{P(E_1,P_3)}=\frac{P(E_1)P(A_2)}{P(P_3|E_1)P(E_1)}=\frac{1\cdot 1/3}{2/3\cdot
    1}=\frac{2}{3}.
\end{align*}
Así, la probabilidad de que al cambiar su elección el concursante encuentre el auto es de $2/3$. 

En cambio, si el concursante mantiene su decisión (la puerta 1), la probabilidad es
$P(A_1|E_1,P_3)=P(A_1)=\frac{1}{3}$. 

Concluimos que $P(A_1|E_1,P_3)=1/3<2/3=P(A_2|E_1,P_3)$, y que al concursante le conviene cambiar su elección.

\item \emph{Calcula la probabilidad de que en un cuarto de 10 personas, al menos 2 cumplan años el
    mismo día. Repite el cálculo para 23, 50 y 75 personas y discute los resultados.}\\
    {\bf Soln.} En general, la probabilidad de que en un grupo de $n$ personas no haya dos que cumplan años el
    mismo día se puede calcular pensando en podemos elegir el cumpleaños para la primera persona de 365 formas
    (cualquier día del año), para la segunda persona sólo tenemos 364 formas, para la tercera 363, etc. Así,
    la probabilidad de que en un grupo de $n$ personas no haya dos que cumplan años el mismo día es:

    \[q_n= \frac{365\cdot 364 \cdot \dots\cdot(365-n)}{365^n}=\frac{365!}{(365-n)!\cdot 365^n}.\]
    Observe que la probabilidad buscada, la de que dos personas en un grupo de $n$ personas sí cumplan años el
    mismo día es exactamente el complemento de $q_n$, i.e., dicha probabilidad es
    \[p_n=1-q_n=1-\frac{365!}{(365-n)!\cdot 365^n}.\]
    Para $n=10,23,50$ y $75$ tenemos que:
    \begin{itemize}
        \item $p_{10}=0.1169$
        \item $p_{23}=0.5073$
        \item $p_{50}=0.9703$
        \item $p_{75}=0.9997$
    \end{itemize}
    
\item \emph{Se tiran 2 dados y se registra el número máximo, ¿cuales son las probabilidades de los eventos
    1, 2, 3, 4, 5, 6?}\\
    {\bf Soln.} Sean $X$ y $Y$ las variables aleatorias que dan el resultado de lanzar el primer y el segundo
    dado, respectivamente. Note que $X$ y $Y$ son independientes, pues los lanzamientos de los dados son
    independientes, y tienen la misma distribución uniforme. Sea $Z$ la variable aleatoria dada como $\max \{X,Y\}$.
    Buscamos $P(Z=i)$ para $i\in\{1,2,3,4,5,6\}$. 
    Primero observemos que en general, para $x\in \mathbb R$,
    \begin{align*}
        P(Z\leq x)=& P(\max\{X,Y\}\leq x)=P(X\leq x,Y\leq x)\stackrel{X,Y\newline ind}{=}P(X\leq x)P(Y\leq
        x)\stackrel{Dist X=Dist Y }{=}P(X\leq x)^2.
    \end{align*}
    Similarmente, $P(Z<x)=(P(X\leq x)-P(X=x))^2$.
    Por lo tanto, 
    \begin{align*}
        P(Z=x)= & P(Z\leq x)-P(Z<x)=P(X\leq x)^2-(P(X\leq x)-P(X=x))^2\\
        =& 2P(X\leq x)P(X=x)-P(X=x)^2.
    \end{align*}
    Para terminar, observe que para $i=1,2,3,4,5,6$, $P(X=i)=1/6$ y $P(X\leq i)=i/6$. Concluimos que
    \begin{itemize}
        \item $P(Z=1)=2\frac{1}{6}\cdot\frac{1}{6}-(\frac{1}{6})^2=\frac{1}{36}$
        \item $P(Z=2)=2\frac{2}{6}\cdot\frac{1}{6}-(\frac{1}{6})^2=\frac{3}{36}$
        \item $P(Z=3)=2\frac{3}{6}\cdot\frac{1}{6}-(\frac{1}{6})^2=\frac{5}{36}$
        \item $P(Z=4)=2\frac{4}{6}\cdot\frac{1}{6}-(\frac{1}{6})^2=\frac{7}{36}$
        \item $P(Z=5)=2\frac{5}{6}\cdot\frac{1}{6}-(\frac{1}{6})^2=\frac{9}{36}$
        \item $P(Z=6)=2\frac{6}{6}\cdot\frac{1}{6}-(\frac{1}{6})^2=\frac{11}{36}$
    \end{itemize}

\item \emph{Prueba que la covarianza de 2 variables independientes es 0.}\\
    {\bf Soln.} La covarianza $Cov(X,Y)$ de las variables aleatorias $X$ y $Y$ se puede calcular como 
    $\mathbb E(XY)-\mathbb E(X)\mathbb E(Y)$.  Por lo tanto, para mostrar que $Cov(X,Y)=0$ basta probar 
    que $\mathbb E(XY)=\mathbb E(X)
    \mathbb E(Y)$. Haremos esto en el caso en que $X$ y $Y$ son variables aleatorias discretas; el caso continuo
    es similar cambiando sumas por integrales. 
    
    Sean $f_X$ y $f_Y$ las funciones de densidad de $X$ y $Y$, respectivamente. Como $X$ y $Y$ son
    independientes, la función de densidad conjunta $f_{X,Y}$ de $X$ y $Y$ cumple que
    $f_{X,Y}(x,y)=f_X(x)f_Y(y)$ para cualesquiera $x$ y $y$. Entonces, efectivamente, 
    \begin{align*}
        \mathbb E(XY) :=& \sum_{x,y} xyf_{X,Y}(x,y) =\sum _{x,y}xyf_X(x)f_Y(y) \\
        = & \sum _x\sum_y xf_X(x)yf_Y(y)=\sum _xxf_X(x)\cdot \sum _yyf_Y(y)= 
        \mathbb E(X)\mathbb E(Y).\hspace{5.8cm} \blacksquare
            \end{align*}
\item \emph{Un alumno decide responder de forma aleatoria un examen en el que cada pregunta tiene
5 opciones de las cuales solo 1 es correcta. Suponiendo que la probabilidad de acertar una
pregunta es de 0.2 y que cada respuesta es independiente de las demás, calcula lo siguiente:}

La probabilidad de acertar es $p=1/5$ y la de no acertar, en consecuencia, es $1-p=4/5$. La calificación de
una pregunta tiene una distribución Bernoulli (prob. de exito $p$); un examen consta de una serie 
de preguntas donde
cada pregunta puede ser contestada correcta o incorrectamente. Por lo tanto, la calificación
(= el número de aciertos) de un examen de $n$ preguntas tiene la distribución binomial 
\[P(Cal=k)=\binom{n}{k}p^k(1-p)^{n-k}.\]
Las preguntas debajo se responden aplicando esta fórmula. 
\begin{itemize} 

\item La probabilidad de tener 10, 20 y 40 aciertos si el examen consiste de 50 preguntas\\
    {\bf Soln.} En este caso $n=50$, y
    \begin{itemize}
        \item si el número de aciertos debe ser 10, $k=10$ y la probabilidad buscada es
            \[P(Cal=10)=\binom{50}{10}\left(\frac{1}{5}\right)^{10}\left(\frac{4}{5}\right)^{40}=0.13981\]
        \item si el número de aciertos debe ser 20, $k=20$ y la probabilidad buscada es
            \[P(Cal=20)=\binom{50}{20}\left(\frac{1}{5}\right)^{20}\left(\frac{4}{5}\right)^{30}=
            0.0006118\]
        \item si el número de aciertos debe ser 40, $k=40$ y la probabilidad buscada es
            \[P(Cal=40)=\binom{50}{40}\left(\frac{1}{5}\right)^{40}\left(\frac{4}{5}\right)^{10}=1.21273\times
            10^{-19}\]
    \end{itemize}
\item La probabilidad de tener 10, 20 y 40 aciertos si el examen consiste de 40 preguntas\\
    {\bf Soln.} En este caso $n=40$, y
    \begin{itemize}
        \item si el número de aciertos debe ser 10, $k=10$ y la probabilidad buscada es
        \[P(Cal=10)=\binom{40}{10}\left(\frac{1}{5}\right)^{10}\left(\frac{4}{5}\right)^{30}=
        0.10745\]
        \item si el número de aciertos debe ser 20, $k=20$ y la probabilidad buscada es
            \[P(Cal=20)=\binom{40}{20}\left(\frac{1}{5}\right)^{20}\left(\frac{4}{5}\right)^{20}=0.00001667\]
        \item si el número de aciertos debe ser 40, $k=40$ y la probabilidad buscada es
            \[P(Cal=40)=\binom{40}{40}\left(\frac{1}{5}\right)^{40}\left(\frac{4}{5}\right)^0=
            1.0995\times 10^{-28}\]
    \end{itemize}
\item La probabilidad de tener 10, 20 y 40 errores si el examen consiste de 40 preguntas\\
    {\bf Soln.} Tener $l$ errores es equivalente a tener $n-l$ aciertos, por lo tanto podemos seguir el
    procedimiento de los casos anteriores. En este caso $n=40$, y
    \begin{itemize}
        \item si el número de errores debe ser 10, $l=10$ y la probabilidad buscada es
            \[P(Cal=40-10)=\binom{40}{30}\left(\frac{1}{5}\right)^{30}\left(\frac{4}{5}\right)^{10}=
            9.773\times 10^{-14}\]
        \item si el número de errores debe ser 20, $l=20$ y la probabilidad buscada es
            \[P(Cal=40-20)=\binom{40}{20}\left(\frac{1}{5}\right)^{20}\left(\frac{4}{5}\right)^{20}=0.00001667\]
        \item si el número de errores debe ser 40, $l=40$ y la probabilidad buscada es
            \[P(Cal=40-40)=\binom{40}{0}\left(\frac{1}{5}\right)^0\left(\frac{4}{5}\right)^{40}=0.000133\]
    \end{itemize}
\end{itemize}

\item \emph{La estatura promedio de los alumnos de una universidad es de 169.83cm, con una desviación estándar
        de 4.5cm. Suponiendo una distribución normal, calcula la probabilidad de que la estatura de un 
        alumno dado:
\begin{itemize}
    \item Se encuentre entre 160cm y 180cm
    \item Sea de al menos 150cm
    \item Sea de máximo 180cm
    \item Sea mayor a 160cm
    \item Sea menor a 190cm
\end{itemize}}
{\bf Soln.} Sea $X$ la variable aleatoria de las estaturas de los alumnos. Por hipótesis sabemos que $X$ tiene
una distribución normal con esperanza $\mu=169.83$ y desviación estándar $\sigma=4.5$. La distribución normal
estándar se obtiene a partir de $X$ como $Z:=\frac{X-\mu}{\sigma}$. Usando las tabla de valores para $Z$,
obtenemos lo siguiente.
\begin{itemize}
    \item La probabilidad de que la estatura de un alumno se encuente entre 160cm y 180cm,  es 
        \[P\left(\frac{160-169.83}{4.5}\leq Z\leq \frac{180-169.83}{4.5}\right)=
        P(-2.184\leq Z\leq 2.260)= 0.9736.\]
        \item La probabilidad de que la estatura de un alumno sea de al menos 150cm,
            \[P\left(Z\geq \frac{150-169.83}{4.5}\right)
            =P(Z\geq -4.4067)= 1. \]
        \item La probabilidad de que la estatura sea de máximo 180cm, es
            \[P\left(Z\leq \frac{180-169.83}{4.5}\right)=P(Z\leq 2.260)= 0.9881. \]
        \item La probabilidad de que la estatura sea mayor a 160cm, es
            \[P\left(Z\geq \frac{160-169.83}{4.5}\right)=P(Z\geq -2.1844)= 0.9855.\] 
        \item La probabilidad de que la estatura de un alumno sea menor a 190cm, es
            \[P\left(Z\leq \frac{190-169.83}{4.5}\right)=P(Z\leq 4.4822)= 1.\] 
\end{itemize}

\item \emph{Una empresa de lámparas incandecentes observa que el número de componentes que fallan
antes de cumplir 100 horas de funcionamiento es una variable aleatoria de Poisson. Si el
número promedio de estos fallos es 8, calcule lo siguiente:}
\begin{itemize}
    \item ¿Cuál es la probabilidad de que falle 1 componente en 25 horas?\\
        {\bf Soln.} Como el promedio en 100 horas es de 8 fallos y suponemos un proceso de Poisson (en
        particular, que el promedio en un intervalo de tiempo es proporcional a la longitud del intervalo),
        en un intervalo de 25 horas el promedio sería de 2. Por lo tanto: 
        \[P(\text{no. de fallos en 25 horas} = 1)= \frac{2^1}{1!}e^{-2}=0.27067. \] 
    \item ¿Cuál es la probabilidad de que fallen no más de 2 componentes en 50 horas?\\
        {\bf Soln.} Análogo al caso de arriba, el promedio de fallos en un intervalo de 50 horas es de 4.
        Por lo tanto:
        \begin{align*}
        P(\text{no. de fallos en 50 horas} \leq 2)=&  
        +P(\text{no. de fallos en 50 horas} = 0)\\
        +P(\text{no. de fallos en 50 horas} =1)
        &+P(\text{no. de fallos en 50 horas} = 2)\\
        &=\frac{4^0}{0!}e^{-4}+\frac{4^1}{1!}e^{-4}+\frac{4^2}{2!}e^{-4}=(1+4+8)e^{-4}=0.23810. \
    \end{align*}
        
    \item ¿Cuál es la probabilidad de que fallen por lo menos 10 en 125 horas?\\
    {\bf Soln.} Como antes, el promedio de fallos en un intervalo de 125 horas es de 10. Por lo tanto:
    \begin{align*}
    &P(\text{no. de fallos en 125 horas} \geq 10)=
    \sum_{k\geq 10}\frac{10^k}{k!}e^{-10}=e^{-10}\sum_{k=0}^\infty \frac{10^k}{k!}-e^{-10}\sum
        _{k=0}^{9}\frac{10^k}{k!}\\
        & = e^{-10}e^{10}-e^{-10}\sum_{k=0}^{9}\frac{10^k}{k!}= 1- \sum
        _{k=0}^{9}\frac{10^k}{k!}=1-0.45793=0.54207.
    \end{align*}
    \end{itemize}
\item \emph{Una pareja tiene dos hijos. Asumiendo que la probabilidad de tener niño o niña es la misma,
        entonces la probabilidad de que tengan un niño y una niña es de $\frac{1}{2}$, mientras que la 
    probabilidad de que tengan 2 niños o 2 niñas es de $\frac{1}{4}$.}
\begin{itemize}
    \item \emph{Suponga que les pregunta si tienen al menos un niño y contestan que sí. ¿Cuál es 
            la probabilidad de que tengan una niña?}\\
    {\bf Soln.} Tenemos en general las siguientes posibilidades para la pareja de hijos: $A$ que ambos sean
    niños, $B$ que el mayor sea niño y el menor niña, $C$ que el mayor sea niña y que el menor sea niño y $D$
    que ambos sean niñas. Sin más información, la probabibilidad de que la pareja tenga un niño y una niña es
    de $1/2$. Sin embargo, una vez que responden que tienen definitivamente un niño, de descarta
    inmediatamente el evento $D$. Así, la probabilidad actualizada de tener una niña es: 
    \[P(\text{tener una niña}|\text{se tiene un niño})=\frac{P(\{B,C\})}{P(\{A,B,C\})}=\frac{2}{3}.\]
    Observe que antes de saber que tenían un niño la probabilidad de que tuvieran una niña era de $3/4$.
    
    \item \emph{Suponga que ve a uno de los hijos y es un niño. ¿Cuál es la probabilidad de que el otro
        hijo sea niña?}
        {\bf Soln.}  Sin mayor precisión, esta pregunta es idéntica a la anterior, en el sentido de que ver
        que uno de los hijos es niño da la misma información (a saber, que la pareja tiene al menos un 
        niño) que el hecho de que la pareja nos informe que tienen al menos un hijo. 

        En cambio, esta pregunta se puede hacer diferente si se asume que hay cierta probabilidad asociada a
        qué hijo se observa. Aquí hay opciones: podríamos decir que es más probable ver a un niño que a una
        niña, o que es más probable ver al hijo mayor que al menor, etc. Al momento, la pregunta no es
        suficientemente clara en lo que solicita.
\end{itemize}
\item \emph{Una baraja ordinaria de 52 cartas se divide aleatoriamente en 4 pilas de 13 cartas cada una.
    Calcule la probabilidad de que cada pila tenga exactamente 1 As.}\\
    {\bf Soln.} Sea $A$ el evento ``cada una de las cuatro pilas tiene un As''. De forma directa tenemos que
    \[P(A)=\frac{\text{no. de particiones en las que cada una de las 4 pilas tiene un As}}
        {\text{no. de posibles particiones en 4 pilas}}\]
    Se obtiene:
    \begin{align*}
        P(A)=\frac{4\binom{48}{12}\cdot 3\binom{36}{12}\cdot 2 \binom{24}{12}\cdot 1\binom{12}{12}}
        {\binom{52}{13}\cdot \binom{39}{13}\cdot \binom{26}{13}\cdot \binom {13}{13}} = \frac{24\cdot
        48!\cdot (13!)^4}{52!\cdot (12!)^4}=\frac{24\cdot 13^4}{52\cdot 51\cdot 50\cdot 49}=0.10549.
    \end{align*}
    
    Una forma más elegante de solucionar el problema es como sigue.
    Para $i\in \{1,2,3,4\}$ sea $A_i$ el evento ``la pila número $i$ tiene exactamente un As''. Buscamos
    $P(A_1\cap A_2\cap A_3\cap A_4)$. Tenemos que
    \begin{align*}
        P(A_1\cap A_2\cap A_3\cap A_4)  = & P(A_1|A_2\cap A_3\cap A_4)\cdot P(A_2|A_3\cap A_4)\cdot
        P(A_3|A_4)\cdot P(A_4)
    \end{align*}
\item \emph{En una determinada etapa de una investigación criminal el inspector al mando está 60 \%
convencido de la culpabilidad de un sospechoso. Supongamos, sin embargo, que una nueva
pieza de evidencia muestra que el delincuente tiene una cierta característica (como la calvicie
o el pelo castaño). Si el 20 por ciento de la población posee esta característica, ¿cuál es la
certeza de la culpabilidad del sospechoso que tiene el inspector si resulta que el sospechoso
tiene la característica?}\\
{\bf Soln.} Buscamos ver si la probabilidad de ser culpable cambia dado que el sospechoso tiene la
característica. Sea $C$ el evento ``el sospechoso es culpable'', $NC$ el evento ``el sospechoso no es
culpable'' y $A$ el evento ``el sospechoso tiene la característica''. Deseamos conocer $P(C|A)$. 
Sabemos que $P(A)=0.20$ y que $P(C)=0.60$. También, dado que estamos suponiendo que el sospechoso tiene 
la característica tenemos que $P(A|C)=1$.  Note que si el sospechoso es de hecho inocente podemos decir
que la probabilidad de que tenga la característica es la general para toda la población, i.e. $P(A|NC)=0.20$.
Por lo tanto,
\[P(C|A)=\frac{P(A|C)P(C)}{P(A|C)P(C)+P(A|NC)P(NC)}=\frac{0.60}{0.60+0.20\cdot 0.4}=0.8823  \]

Concluimos que el inspector puede mejorar su convicción de que el sospechoso es culpable a un 88\%.
\end{enumerate}
\end{document}
